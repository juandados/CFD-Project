
\documentclass[final]{beamer}

\usepackage[scale=1.24]{beamerposter} % Use the beamerposter package for laying out the poster

\usetheme{confposter} % Use the confposter theme supplied with this template

\setbeamercolor{block title}{fg=ngreen,bg=white} % Colors of the block titles
\setbeamercolor{block body}{fg=black,bg=white} % Colors of the body of blocks
\setbeamercolor{block alerted title}{fg=white,bg=dblue!70} % Colors of the highlighted block titles
\setbeamercolor{block alerted body}{fg=black,bg=dblue!10} % Colors of the body of highlighted blocks
% Many more colors are available for use in beamerthemeconfposter.sty

%-----------------------------------------------------------
% Define the column widths and overall poster size
% To set effective sepwid, onecolwid and twocolwid values, first choose how many columns you want and how much separation you want between columns
% In this template, the separation width chosen is 0.024 of the paper width and a 4-column layout
% onecolwid should therefore be (1-(# of columns+1)*sepwid)/# of columns e.g. (1-(4+1)*0.024)/4 = 0.22
% Set twocolwid to be (2*onecolwid)+sepwid = 0.464
% Set threecolwid to be (3*onecolwid)+2*sepwid = 0.708

\newlength{\sepwid}
\newlength{\onecolwid}
\newlength{\twocolwid}
\newlength{\threecolwid}
\setlength{\paperwidth}{48in} % A0 width: 46.8in
\setlength{\paperheight}{36in} % A0 height: 33.1in
\setlength{\sepwid}{0.024\paperwidth} % Separation width (white space) between columns
\setlength{\onecolwid}{0.22\paperwidth} % Width of one column
\setlength{\twocolwid}{0.464\paperwidth} % Width of two columns
\setlength{\threecolwid}{0.708\paperwidth} % Width of three columns
\setlength{\topmargin}{-0.5in} % Reduce the top margin size
%-----------------------------------------------------------

\usepackage{graphicx}  % Required for including images

\usepackage{booktabs} % Top and bottom rules for tables

%----------------------------------------------------------------------------------------
%	TITLE SECTION 
%----------------------------------------------------------------------------------------

\title{Plankton Barbecue} % Poster title

\author{Juan D. Chacon} % Author(s)

\institute{Simon Fraser University Spring 2019} % Institution(s)

%----------------------------------------------------------------------------------------

\begin{document}

\addtobeamertemplate{block end}{}{\vspace*{2ex}} % White space under blocks
\addtobeamertemplate{block alerted end}{}{\vspace*{2ex}} % White space under highlighted (alert) blocks

\setlength{\belowcaptionskip}{2ex} % White space under figures
\setlength\belowdisplayshortskip{2ex} % White space under equations

\begin{frame}[t] % The whole poster is enclosed in one beamer frame

\begin{columns}[t] % The whole poster consists of three major columns, the second of which is split into two columns twice - the [t] option aligns each column's content to the top

\begin{column}{\sepwid}\end{column} % Empty spacer column

\begin{column}{\onecolwid} % The first column

%------------------------------------------------
%	Planton Picture
%------------------------------------------------
\begin{figure}
\includegraphics[width=0.45\linewidth]{images/planktonBBQRCropped.pdf}
\end{figure}

%----------------------------------------------------------------------------------------
%	Introduction
%----------------------------------------------------------------------------------------

\begin{block}{Introduction}

\textbf{Natural Thermal Convection}

Those plumes comming out from our BBQ grill are caused by a differences on density consequence of temperature varitions.
\vspace{1em}
\begin{alertblock}{Governing Equations}
\begin{itemize}
\item \textbf{($\approx$) Momentum:}
$$\frac{\partial\omega}{\partial t}+\frac{\partial\psi}{\partial y}\frac{\partial\omega}{\partial x}-\frac{\partial\psi}{\partial x}\frac{\partial\omega}{\partial y}=\frac{1}{Re}\nabla^{2}\omega+\beta g\frac{\partial T}{\partial x}$$
\item \textbf{Stream fnc - Vorticity - Velocity:}
$$\nabla^{2}\psi=-\omega,\:\: u=\frac{\partial\psi}{\partial y},\:\: v=-\frac{\partial\psi}{\partial x}$$
\item \textbf{Energy:}
$$\frac{\partial T}{\partial t}+\left(\overrightarrow{u}\cdot\nabla T\right)=\frac{1}{RePr}\nabla^{2}T$$
\item \textbf{(Bonus!!!) Mass Transfer:}
$$\frac{\partial c}{\partial t}+\left(\overrightarrow{u}\cdot\nabla c\right)=\frac{1}{Pe}\nabla^{2}c$$
\end{itemize}
\end{alertblock}
The last Eq. models a substance moving with the fluid.
\end{block}

%----------------------------------------------------------------------------------------
%	METHOD
%----------------------------------------------------------------------------------------
\vspace{-1em}
\begin{block}{Method}
To solve numerically the equations:
\begin{enumerate}
\item Set up the initial conditions for $\psi$, $u$ and $v$.
\item Solve the vorticity equation. (Exp FTCS).
\item Solve the Poisson equation for $\psi$. (Std. discr. $\nabla^{2}$).
\item Compute the new velocity from $\psi$.
\item Solve the Energy equation for $T$. (Exp FTCS).
\item Solve the Mass transfer equation. (Exp FTCS).
\item Go to step 2.
\end{enumerate}
\end{block}

\end{column} % End of the first column

\begin{column}{\sepwid}\end{column} % Empty spacer column

\begin{column}{\twocolwid} % Begin a column which is two columns wide (column 2)

\begin{columns}[t,totalwidth=\twocolwid] % Split up the two columns wide column

\begin{column}{\onecolwid}\vspace{-.6in} % The first column within column 2 (column 2.1)

%----------------------------------------------------------------------------------------
%	COL2
%----------------------------------------------------------------------------------------

\begin{block}{BBQ Grill: Air Thermal Convection}
As benchmark paramenters we chose:
$$Re=4365.30, \: Pr=0.72, \: Pe=0.07$$
For a rectangular Cavity width dimensions: 0.65m height, and 2.6m wide, and normalized temperatures Tc=-0.5, Th=0.5. As shown in the figure the fluid shows some cell formations at long term.
\end{block}

%----------------------------------------------------------------------------------------

\end{column} % End of column 2.1

\begin{column}{\onecolwid}\vspace{-.6in} % The second column within column 2 (column 2.2)

%----------------------------------------------------------------------------------------
%	P
%----------------------------------------------------------------------------------------

\begin{block}{Cells and Spatial grid Dependence}
1. Factorise $x^2-x-12$.
\begin{figure}
\includegraphics[width=1.0\linewidth]{images/cells.pdf}
\caption{Cell behaviour}
\end{figure}
\end{block}

%----------------------------------------------------------------------------------------

\end{column} % End of column 2.2

\end{columns} % End of the split of column 2 - any content after this will now take up 2 columns width

%----------------------------------------------------------------------------------------
%	IMPORTANT To REMEMBER
%----------------------------------------------------------------------------------------

\begin{figure}
\includegraphics[width=1.0\linewidth]{images/figure1.pdf}
\caption{Cell behaviour}
\end{figure}
%----------------------------------------------------------------------------------------

\begin{columns}[t,totalwidth=\twocolwid] % Split up the two columns wide column again

\begin{column}{\onecolwid} % The first column within column 2 (column 2.1)

%----------------------------------------------------------------------------------------
%	EXAMPLE OF FACTORISATION
%----------------------------------------------------------------------------------------

\begin{block}{Numerical Convergence}
\begin{table}
\begin{tabular}{|c|c|c|c|c|c|}
\hline 
\multicolumn{6}{|c|}{\textbf{Spatial Convergence order}}\tabularnewline
\hline 
$u$ & 1.5339 & $v$ & 1.3119 & $T$ & 0.3145\tabularnewline
\hline 
\end{tabular}
\end{table}
\end{block}

%----------------------------------------------------------------------------------------

\end{column} % End of column 2.1

\begin{column}{\onecolwid} % The second column within column 2 (column 2.2)

%----------------------------------------------------------------------------------------
%	PROOF OF VIETA'S FORMULAS
%----------------------------------------------------------------------------------------

\begin{block}{ Proof of Vieta's Formulas}
The same we could do with another pattern, which state that $x_1 x_2 = \frac{c}{a}$, but proving this is going to be your task in next section.

\end{block}

%----------------------------------------------------------------------------------------

\end{column} % End of column 2.2

\end{columns} % End of the split of column 2

\end{column} % End of the second column

\begin{column}{\sepwid}\end{column} % Empty spacer column

\begin{column}{\onecolwid} % The third column

%----------------------------------------------------------------------------------------
%	CONCLUSION
%----------------------------------------------------------------------------------------

\begin{block}{And the plankton?}
\textbf{}
	Pythoplackton sinks in the ocean (passive swimmers), 
the temperature in the ocean varies upto 10C in a year,
Is the convection what have this important individuals
\begin{figure}
\includegraphics[width=1.0\linewidth]{images/planktonSim.pdf}
\caption{Cell behaviour}
\end{figure}

\end{block}


%----------------------------------------------------------------------------------------
%	ACKNOWLEDGEMENTS
%----------------------------------------------------------------------------------------

\setbeamercolor{block alerted title}{fg=black,bg=nyellow} % Change the alert block title colors
\setbeamercolor{block alerted body}{fg=black,bg=white} % Change the alert block body colors

\begin{alertblock}{Final Thoughts!}
\textbf{Conclusions}
\begin{itemize}
\item The number of cells depends on the geometry, aspect ratio, as well as the fluid parameters.
\item The evidence of thermal cells depends on the size of the spatial discretization.
\item (n.) factor $\rightarrow$ two multiplied factors give result
\end{itemize}
\textbf{Pendings}
\begin{itemize}
\item A solver using the projection method was considered, however, the stagered grid was an issue for the energy equation.
\item An implentation using the second order FTBS scheme will help to deal with higher Reynolds numbers.
\item The time convergence order might be improved using Crank Nicholson.
\end{itemize}

\end{alertblock}
%----------------------------------------------------------------------------------------

\end{column} % End of the third column

\end{columns} % End of all the columns in the poster

\end{frame} % End of the enclosing frame

\end{document}
